\documentclass[usenames,dvipsnames]{beamer}
\usepackage{beamerthemeshadow}
\setbeamertemplate{headline}{}
\setbeamertemplate{footline}{}
\usepackage{amsmath}
\usepackage{amsfonts}
\usepackage{bbold}
\usepackage{graphicx}
\usepackage{amssymb}
\usepackage{tikz}
\usepackage[T1]{fontenc}
\usepackage{setspace}
\usepackage[utf8]{inputenc}
\usepackage[french]{babel}
\usepackage{pgffor}
\usepackage{listings}
\usepackage{stmaryrd}
\usepackage{caption}
\usepackage{algorithm2e}

\title{Segmentation d'image par marches aléatoires}
\date{}

\begin{document}

\frame{\titlepage}

\section{I. Présentation du problème}
\frame{\frametitle{Problème}

  La \textbf{segmentation d'image} est un problème de vision par ordinateur qui consiste à \textbf{partitionner} une image donnée en différentes régions selon des \textbf{critères} prédéfinis.

  \medskip{}
  On s'intéresse ici au cas où \textbf{certains pixels sont étiquetés}, on cherche alors à attribuer une étiquette aux autres pixels.

  On supposera de plus que les \textbf{objets} sont \textbf{homogènes} et \textbf{tous étiquetés}.

  \bigskip
  \begin{figure}
    \begin{tikzpicture}
      \node[left] (test) at (0,0) {\includegraphics[width=3.5cm]{images/n_test}};
      \node[left] (test) at (0,0) {\includegraphics[width=3.5cm]{images/stest}};
      \draw[->] (0.5,0) -- (1.5,0);
      \node[right] (test) at (2,0) {\includegraphics[width=3.5cm]{images/sptest}};
    \end{tikzpicture}
  \end{figure}
}


\frame{\frametitle{Solution}
  On va faire des \textbf{marches aléatoires} partant de chacun des pixels, et on retient l'étiquette du pixel sur lequel on retombe le plus souvent en premier.

  \bigskip
  \begin{figure}
    \begin{tikzpicture}
      \node (test) at (0,0) {\includegraphics[width=5cm]{rand}};
      \node (test) at (0,0) {\includegraphics[width=5cm]{srand}};
      \node (test) at (0,0) {\includegraphics[width=5cm]{rand1}};
      \node (test) at (0,0) {\includegraphics[width=5cm]{rand2}};
      \node (test) at (0,0) {\includegraphics[width=5cm]{rand3}};
      \draw[fill=yellow] (5cm*44/512,-5cm*44/512) circle (.05);
    \end{tikzpicture}
  \end{figure}
}


\frame{\frametitle{Graphe}
  \begin{align*}
    V &= \llbracket 0, H - 1 \rrbracket \times \llbracket 0, L - 1 \rrbracket \\
    E &= \{\,\{v_i,v_j\} \subseteq V \quad|\quad \Vert v_i-v_j\Vert = 1\,\} \\
      &\forall i,j \in V,\,
        \begin{cases}
          \omega_{ij} \in \mathbb{R}_+^* & \mbox{si } e_{ij} \in E \quad(\mbox{p.ex. } e^{-\beta\lVert p_i-p_j \rVert ^2}) \\
          \omega_{ij} = 0 & \mbox{sinon}
        \end{cases}
  \end{align*}

  
  \begin{figure}
    \begin{tikzpicture}
      \filldraw[red] (1,1) rectangle (2,2);
      \filldraw[BurntOrange] (1,0) rectangle (2,1);
      \filldraw[cyan] (0,1) rectangle (1,2);
      \filldraw[red] (1,2) rectangle (2,3);
      \filldraw[magenta] (2,1) rectangle (3,2);
      \filldraw[Peach] (2,0) rectangle (3,1);
      \filldraw[brown] (0,0) rectangle (1,1);
      \filldraw[DarkOrchid] (2,2) rectangle (3,3);
      \draw (0,0) grid (3,3);
      \filldraw[blue] (0,2) rectangle (1,3);
      \draw[line width = 1.2mm] (1.5,1.5) -- (1.5,2.5);
      \draw[line width = .9mm] (1.5,1.5) -- (2.5,1.5);
      \draw[line width = .7mm] (1.5,2.5) -- (2.5,2.5);
      \draw[line width = 1mm] (2.5,2.5) -- (2.5,1.5);
      \draw[line width = .7mm] (1.5,1.5) -- (1.5,.5);
      \draw[line width = 1.1mm] (1.5,.5) -- (2.5,.5);
      \draw[line width = .6mm] (2.5,1.5) -- (2.5,.5);
      \draw[line width = .8mm] (.5,.5) -- (1.5,.5);
      \draw[line width = .8mm] (.5,2.5) -- (.5,1.5);
      \draw[line width = .3mm] (.5,1.5) -- (.5,.5);
      \draw[line width = .1mm] (.5,2.5) -- (1.5,2.5);
      \draw[line width = .1mm] (.5,1.5) -- (1.5,1.5);
      \draw[fill=white] (.5,.5) circle (.15);
      \draw[fill=white] (.5,1.5) circle (.15);
      \draw[fill=white] (.5,2.5) circle (.15);
      \draw[fill=white] (1.5,.5) circle (.15);
      \draw[fill=white] (1.5,1.5) circle (.15);
      \draw[fill=white] (1.5,2.5) circle (.15);
      \draw[fill=white] (2.5,.5) circle (.15);
      \draw[fill=white] (2.5,1.5) circle (.15);
      \draw[fill=white] (2.5,2.5) circle (.15);
    \end{tikzpicture}
  \end{figure}
}


\frame{\frametitle{Marches aléatoires}
  \begin{gather*}
    \forall i \in V,\quad d_i = \sum_{e_{ij} \in E}\omega_{ij} \\
    \forall (i,j) \in V^2,\quad p(i \to j) = \frac{\omega_{ij}}{d_i}
  \end{gather*}


  \begin{figure}
    \begin{tikzpicture}
      \filldraw[red] (1,1) rectangle (2,2);
      \filldraw[BurntOrange] (1,0) rectangle (2,1);
      \filldraw[cyan] (0,1) rectangle (1,2);
      \filldraw[red] (1,2) rectangle (2,3);
      \filldraw[magenta] (2,1) rectangle (3,2);
      \filldraw[Peach] (2,0) rectangle (3,1);
      \filldraw[brown] (0,0) rectangle (1,1);
      \filldraw[DarkOrchid] (2,2) rectangle (3,3);
      \filldraw[blue] (0,2) rectangle (1,3);
      \draw (0,0) grid (3,3);
      \draw[->] (1.5,1.5) -- (1.5,0.7);
      \draw[->] (1.5,1.5) -- (1.5,2.7);
      \draw[->] (1.5,1.5) -- (2.5,1.5);
      \draw[->] (1.5,1.5) -- (1.3,1.5);
    \end{tikzpicture}
  \end{figure}
}


\frame{\frametitle{Marches aléatoires}
  \[x_i^s = \mathbb{P} (\mbox{"la marche aléatoire partant de $i$ arrive en premier en $S$"})\]
  
  Alors \[\forall i \in V_{NE}, \forall s \in S,\quad
    x_i^s = \frac{\sum\limits_{e_{ij} \in E}{\omega_{ij} x_j^s}}{d_i}\]

  On cherche $x$ avec comme conditions aux bords
  \[\forall i \in V_E, \forall s \in S,\quad x_i^s = \begin{cases} 1 & \mbox{si $i$ est étiqueté par $s$} \\ 0 & \mbox{sinon} \end{cases} \]
}


\frame{\frametitle{Algorithme : Automate cellulaire}
  \begin{columns}
    \begin{column}{0.57\textwidth}
      \small
      \begin{algorithm}[H]
        \SetAlgoLined
        \SetKwIF{If}{ElseIf}{Else}{si}{:}{sinon si}{sinon:}{}
        \SetKwFor{ForEach}{pour chaque}{faire}{fin}
        \SetKwFor{For}{pour}{faire}{fin}
        \SetKw{From}{de}
        \SetKw{To}{à}
        \SetKw{In}{dans}
        \SetKwProg{Fn}{Fonction}{:}{}
        \SetKwInOut{Input}{entrée}
        \SetKwInOut{Output}{sortie}
        \SetKw{Return}{renvoie}

        \SetKwData{Solutions}{solutions}
        \SetKwData{State}{état}
        \SetKwData{Board}{échiquier}
        \SetKwFunction{Rec}{rec}
        \SetKwData{Pos}{$p$}
        \SetKwData{k}{$k$}

        \Input{graphe, nombre $n$ d'itérations}
        \Output{probabilités}

        $x$ $\gets$ répartition approximative avec conditions aux bords\;
        \For{$k$ \From 1 \To $n$}{
          \ForEach{$i$ \In $V_{NE}$}{
            \ForEach{$s$ \In $S$}{
              $\underline{x_i^s} = \frac{\sum_{e_{ij} \in E}{\omega_{ij}\, x_j^s}}{d_i}$ \;
            }
          }
          $x \gets \underline{x}$ \;
        }
        \Return{$x$}\;
        \normalsize
      \end{algorithm}
    \end{column}
    \begin{column}{0.45\textwidth}
      \begin{figure}
        \begin{tikzpicture}
          \node[above] at (.9,.5) {\includegraphics[height=1.5cm]{a}};
          \node[below] at (.9,.5) {image};
          \node[above] at (2.7,.5) {\includegraphics[height=1.5cm]{sa}};
          \node[below] at (2.7,.5) {étiquettes};
          
          \node[above] at (0,-2.2) {\includegraphics[height=1.5cm]{cell1}};
          \node[below] at (0,-2.2) {0};
          \node[above] at (1.8,-2.2) {\includegraphics[height=1.5cm]{cell2}};
          \node[below] at (1.8,-2.2) {50};
          \node[above] at (3.6,-2.2) {\includegraphics[height=1.5cm]{cell3}};
          \node[below] at (3.6,-2.2) {300};
          \node[above] at (.9,-4.4) {\includegraphics[height=1.5cm]{cell4}};
          \node[below] at (.9,-4.4) {1000};
          \node[above] at (2.7,-4.4) {\includegraphics[height=1.5cm]{cell5}};
          \node[below] at (2.7,-4.4) {10000};
        \end{tikzpicture}
      \end{figure}
    \end{column}
  \end{columns}
}


\frame{\frametitle{Résolution analytique}
  \begin{block}{Proposition}
    Les valeurs prises par une fonction harmonique sont entre les extrema des conditions aux bords.
  \end{block}
  \begin{block}{Corollaire}
    Les fonctions harmoniques sont uniques pour des conditions aux bords fixées.
  \end{block}
}


\frame{\frametitle{Résolution analytique}  
  \textbf{Preuve} :

  Soit $x_i^s$ un maximum sur $V_{NE}$.

  Si $i$ est dans l'intérieur de $V_{NE}$, $x_i^s = \frac{\sum_{e_{ij}}{\omega_{ij}\,x_j^s}}{d_i}$ avec les $j$ dans $V_{NE}$ donc $x_j^s = x_i^s$.

  On se ramène donc au cas où $i$ est sur les bords de $V_{NE}$, $x_i^s = \frac{\sum_{e_{ij}}{\omega_{ij}\,x_j^s}}{d_i} + \frac{\sum_{e_{ik}}{\omega_{ik}\,x_k^s}}{d_i}$ avec les $j$ dans $V_{NE}$ et les $k$ dans $V_E$. Alors il y a un $k$ tel que $x_k^s >= x_i^s$.

  Idem si $x_i^s$ est un minimum.

  \bigskip
  \textbf{Preuve du Corollaire} :

  Soit $x$ et $y$ deux fonctions harmoniques avec mêmes conditions aux bords. Alors la fonction $(x - y)$ est aussi harmonique avec 0 comme condition aux bords, donc $x=y$.

}


\frame{\frametitle{Résolution analytique}
  \[
    L_{ij} =
    \begin{cases}
      d_i & \mbox{si } i=j \\
      -\omega_{ij} & \mbox{sinon}
    \end{cases}\]

  \bigskip
  \begin{block}{Proposition}
    La solution $x$ au problème est la fonction qui minimise $^txLx$.
  \end{block}
}


\frame{\frametitle{Résolution analytique}
  \[^txLx = \sum_{i \in V}{x_i (x_i d_i - \sum_{j \in V}{\omega_{ij}x_j})} = \sum_{e_{ij} \in E}{\omega_{ij}(x_i-x_j)^2}\]
  Donc $^txLx$ est convexe et $^txLx \xrightarrow[x \to\infty]{} \infty$ donc pour toutes conditions aux bords il existe un minimum atteint pour $x$.

  Soit $i \in V_{NE}$, on isole le terme
  \begin{align*}
    & \sum_{e_{ij} \in E}{\omega_{ij}(x_i-x_j)} \\
    = & \lVert X_j - x_i \mathbb{1} \rVert ^2
        \qquad\mbox{avec } <a,b> = \sum_{e_{ij} \in E}{\omega_{ij}a_ib_j} \\
    = & \lVert X_j \rVert ^2 - 2x_i<X_j, \mathbb{1} > + x_i^2 \lVert \mathbb{1} \rVert ^2
  \end{align*}

  La somme est minimale donc \[x_i = \frac{\sum_{e_{ij} \in E}{\omega_{ij}\,x_j}}{d_i}\]
}


\frame{\frametitle{Résolution analytique}
  \[
    L = \begin{bmatrix} L_E & B \\ ^t B & L_{NE} \end{bmatrix},\quad
    x = \begin{bmatrix} x_E \\ x_{NE} \end{bmatrix}
  \]

  \[(Lx)_i = x_i d_i - \sum_{e_{ij} \in E}{\omega_{ij}x_j}\]
  Donc
  \[Lx = \begin{bmatrix} ? \\ 0 \end{bmatrix}\]
  Donc
  \[L_{NE}\,x_{NE} = -^tB\,x_E\]

  $\implies$ Résolution de $(\left\vert S \right\vert-1)$ sytèmes linéaires creux de grande taille (5 coefficients non nuls par ligne/colonne dans $L$), le dernier s'obtenant à partir des autres, $x_i^{\left\vert S \right\vert} = 1 - \sum_{s=1}^{\left\vert S \right\vert - 1}{x_i^s}$.
}


\frame{\frametitle{Exemples}
  \begin{figure}
    \begin{tikzpicture}
      \node at (0,0) {\includegraphics[height=3cm]{lines}};
      \node at (0,0) {\includegraphics[height=3cm]{slines}};
      \draw (-1.5,-1.5) rectangle (1.5,1.5);
      \draw[->] (2,0) -- (3,0);
      \node at (5,0) {\includegraphics[height=3cm]{plines}};
      \draw (3.5,-1.5) rectangle (6.5,1.5);
      
      \draw[fill=black] (-3,-3) rectangle (-1,-5);      
      \node at (-2,-4) {\includegraphics[height=2cm]{plines0}};
      \draw[fill=black] (0,-3) rectangle (2,-5);
      \node at (1,-4) {\includegraphics[height=2cm]{plines1}};
      \draw[fill=black] (3,-3) rectangle (5,-5);
      \node at (4,-4) {\includegraphics[height=2cm]{plines2}};
      \draw[fill=black] (6,-3) rectangle (8,-5);
      \node at (7,-4) {\includegraphics[height=2cm]{plines3}};
    \end{tikzpicture}
  \end{figure}
}


\frame{\frametitle{Exemples}
  \begin{figure}
    \begin{tikzpicture}
      \node at (0,0) {\includegraphics[height=3cm]{n_noise}};
      \node at (0,0) {\includegraphics[height=3cm]{sn_noise}};
      \draw (-1.5,-1.5) rectangle (1.5,1.5);
      \draw[->] (2,0) -- (3,0);
      \node at (5,0) {\includegraphics[height=3cm]{pnoise}};
      \draw (3.5,-1.5) rectangle (6.5,1.5);

      \node at (0,-4) {\includegraphics[height=3cm]{n_bw}};
      \node at (0,-4) {\includegraphics[height=3cm]{sbw}};
      \draw (-1.5,-5.5) rectangle (1.5,-2.5);
      \draw[->] (2,-4) -- (3,-4);
      \node at (5,-4) {\includegraphics[height=3cm]{pbw}};
      \draw (3.5,-5.5) rectangle (6.5,-2.5);
    \end{tikzpicture}
  \end{figure}
}


\frame{\frametitle{Extension}
  L'algorithme est très flexible :
  \begin{itemize}
  \item graphes : s'adapte à toutes sortes d'entrées (ex: images médicales 3D, classification de donées...).
  \item la fonction heuristique $\omega$ peut être modifiée selon les besoins.
  \end{itemize}

  \textbf{Exemple} :

  On peut implémenter une fonction $\omega$ qui prend en compte les différentes couleurs de chaque object.
  
  On considère le graphe complet RGB, sur lequel les poids sont initialisés selon la distance euclidienne. Puis on annule les arrêtes reliant deux pixels étiquetés par la même étiquette. Pour finir on applique Floyd-Warshall au graphe ainsi obtenu pour définir une nouvelle mesure de distance.
}


\frame{\frametitle{Extension}
  \begin{tikzpicture}
    \node at (0,0) {\includegraphics[height=3cm]{stripes2}};
    \draw (-1.5,-1.5) rectangle (1.5,1.5);
    \node at (3.5,0) {\includegraphics[height=3cm]{stripes2}};
    \node at (3.5,0) {\includegraphics[height=3cm]{sstripes2}};
    \draw (2,-1.5) rectangle (5,1.5);
    \node at (7,0) {\includegraphics[height=3cm]{pstripes2}};
    \draw (5.5,-1.5) rectangle (8.5,1.5);
    
    \node at (0,-4) {\includegraphics[height=3cm]{pen}};
    \draw (-1.5,-5.5) rectangle (1.5,-2.5);
    \node at (3.5,-4) {\includegraphics[height=3cm]{pen}};
    \node at (3.5,-4) {\includegraphics[height=3cm]{spen}};
    \draw (2,-5.5) rectangle (5,-2.5);
    \node at (7,-4) {\includegraphics[height=3cm]{ppen}};
    \draw (5.5,-5.5) rectangle (8.5,-2.5);
  \end{tikzpicture}
}


\frame{\frametitle{Bibliographie}
  \begin{thebibliography}{9}
  \bibitem{rwmodernintro}
    Gregory F. Lawler and Vlada Limic :
    \textit{Random Walk: A Modern Introduction}
    (2010)

  \bibitem{rwforimgseg}
    Leo Grady :
    Random Walks for Image Segmentation, \textit{IEEE Trans. on Pattern Analysis and Machine Intelligence}, Vol. 28, No. 11, pp. 1768-1783, Nov., 2006.

  \bibitem{rwelecnet}
    Peter G. Doyle and J. Laurie Snell :
    \textit{Random walks and electric networks}
    (1984)
  \end{thebibliography}
}


\end{document}
%%% Local Variables:
%%% mode: latex
%%% TeX-master: t
%%% End:
