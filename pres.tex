\documentclass{beamer}
\usepackage{beamerthemeshadow}
\setbeamertemplate{headline}{}
\setbeamertemplate{footline}{}
\usepackage{amsmath}
\usepackage{amsfonts}
\usepackage{graphicx}
\usepackage{amssymb}
\usepackage{tikz}
\usepackage[T1]{fontenc}
\usepackage{setspace}
\usepackage[utf8]{inputenc}
\usepackage[french]{babel}
\usepackage{pgffor}
\usepackage{listings}
\usepackage{stmaryrd}

\title{Segmentation d'image par marches aléatoires}
\date{}

\begin{document}

\frame{\titlepage}

\section{I. Présentation du problème}
\frame{\frametitle{Problème}

  La \textbf{segmentation d'image} est un problème de vision par ordinateur qui consiste à \textbf{partitionner} une image donnée en différentes régions selon des \textbf{critères} prédéfinis.

  \medskip{}
  On s'intéresse ici au cas où \textbf{certains pixels sont étiquetés}, on cherche alors à attribuer une étiquette aux autres pixels.

  On supposera de plus que les \textbf{objets} sont \textbf{homogènes} et \textbf{tous étiquetés}.

  \bigskip
  \begin{center}
    \begin{tikzpicture}
      \node[left] (test) at (0,0) {\includegraphics[width=3.5cm]{images/n_test}};
      \node[left] (test) at (0,0) {\includegraphics[width=3.5cm]{images/stest}};
      \draw[->] (0.5,0) -- (1.5,0);
      \node[right] (test) at (2,0) {\includegraphics[width=3.5cm]{images/sptest}};
    \end{tikzpicture}
  \end{center}
}


\frame{\frametitle{Solution}
  On va faire des \textbf{marches aléatoires} partant de chacun des pixels, et on retient l'étiquette du pixel sur lequel on retombe le plus souvent en premier.
}

\frame{\frametitle{Marches aléatoires}
  On définit un graphe simple pondéré par
  \begin{gather*}
    V = \llbracket 0, H - 1 \rrbracket \times \llbracket 0, L - 1 \rrbracket \\
    E = \{\,\{v_i,v_j\} \subseteq V \quad|\quad \Vert v_i-v_j\Vert = 1\,\} \\
    \forall e_{ij} \in E,\, \omega_{ij} \in \mathbb{R}_+^*\quad, \quad
    d_i = \sum_{e_{ij} \in E}\omega_{ij}
  \end{gather*}

  $\mbox{et } (S_n) \mbox{ une marche aléatoire telle que } \mathbb{P}[S_{n+1}=v_j | S_n=v_i] = \frac{\omega_{ij}}{d_i}$


  \begin{center}
    \begin{tikzpicture}
      \filldraw[red] (1,1) rectangle (2,2);
      \filldraw[orange] (1,0) rectangle (2,1);
      \filldraw[green] (0,1) rectangle (1,2);
      \filldraw[red] (1,2) rectangle (2,3);
      \filldraw[magenta] (2,1) rectangle (3,2);
      \draw (0,0) grid (3,3);
      \draw[->] (1.5,1.5) -- (1.5,0.7);
      \draw[->] (1.5,1.5) -- (1.5,2.7);
      \draw[->] (1.5,1.5) -- (2.5,1.5);
      \draw[->] (1.5,1.5) -- (1.3,1.5);
    \end{tikzpicture}
  \end{center}
}


\frame{\frametitle{t}
  Soit $(S_n^i)$ une marche aléatoire partant de $v_i \in V$, on définit la v.a.
  
  $X_i = $ "étiquette du premier pixel étiqueté atteint"
  
  Alors \[\forall v_i \in V, \forall s \mbox{ étiquette},
  \mathbb{P} [X = s] = \frac{\sum_{e_{ij}}{\omega_{ij}\mathbb{P}[X_j = s]}}{d_i}\]

  \bigskip
  Pour toute étiquette $s$, on cherche $f_s$ vérifiant
  \[\forall v_i \in V,\quad f_s(v_i) = \frac{\sum_{e_{ij}}{\omega_{ij}\ f_s(v_j)}}{d_i}\]
}


\frame{\frametitle{}
  \begin{block}{Propriétés}
    \begin{itemize}
    \item $\forall s$ étiquette, $f_s$ est unique
    \item $\forall v_{ij} \in V ,\quad x_{ij} \in [0,1]$
    \item $f_s$ est harmonique
    \end{itemize}
  \end{block}
}


\frame{\frametitle{}
  \[
    L_{ij} =
    \begin{cases}
      d_{ij} & \mbox{si } i = j \\
      -w_{ij} & \mbox{si } i \neq j \\
      0 & \mbox{sinon}
    \end{cases}
  \]
}


\end{document}
%%% Local Variables:
%%% mode: latex
%%% TeX-master: t
%%% End:
